If scientific computing lives at the Linux command line,
then being a Windows users can be tricky.
The two parts of our workflow that are most difficult to implement on Windows
are executing Makefiles and creating symbolic links.

My real recommendation is to get a Mac or Linux machine.

If a team member is stuck on Windows,
we recommend simply installing Ubuntu on Windows Subsystem for Linux (WSL).
Here are instructions:
\href{https://ubuntu.com/tutorials/install-ubuntu-on-wsl2-on-windows-10}{Install Ubuntu on WSL2 on Windows 10}.
You will need to install the Linux versions of all applications, such as R, LaTeX, and Stata.
This is genuine Ubuntu; it will not invoke your Windows executables.

\subsection{A few details}

\begin{itemize}
\item
The default system shell.
If your team includes a WSL2 user,
that user will need to
change the default non-interactive shell in their Ubuntu installation
or
your team will need to declare \texttt{SHELL=bash} at top of each Makefile.
One of these is necessary \href{https://wiki.ubuntu.com/DashAsBinSh}{because}
``In Ubuntu 6.10, the default system shell, \texttt{/bin/sh}, was changed to dash (the Debian Almquist Shell); previously it had been bash (the GNU Bourne-Again Shell).''
\item
Stata will work on WSL2.
\href{https://www.stata.com/products/linux/}{Stata says} ``Can my copy of Stata run on both my Linux desktop and Windows laptop?
Yes. Stata licenses are not platform specific so you can use your license to install Stata on any of the supported platforms.''
\item
Windows uses different line endings than Mac and Linux.
This is annoying.
CLRF prefs need to be set in \texttt{.gitattributes} or in \texttt{.gitconfig} to set all file types to be LF instead of CRLF.
\end{itemize}
