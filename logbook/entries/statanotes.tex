Stata usage among older economists may just reflect path dependence.
It has a number of shortcomings, such as no proper package mangagement.
Beware of dumb \href{https://ifs.org.uk/docs/stata_gotchasJan2014.pdf}{Stata gotchas}.

\subsubsection{Command-line execution}

For old versions (before Stata 16), a few Stata commands only work in ``interactive'' mode and not when Stata is invoked at the command line.
\begin{itemize}
	\item \texttt{graph export}: 
	``ps and eps are available for all versions of Stata; png and tif are available for all versions of Stata except Stata(console) for Unix; pdf is available only for Stata for Windows and Stata for Mac.''
	See \url{https://www.stata.com/manuals13/g-2graphexport.pdf#g-2graphexport}.
	We usually choose eps format.
\end{itemize}


\subsubsection{Temporary files}
Stata loads at most one dataset at a time, 
thus opening a dataset will cause Stata to discard the dataset that is currently in memory.
This constraint may push you to consider saving lots of intermediate output or temporary files to the \texttt{output} folder of your task.
Don't.
Use Stata's \texttt{tempfile} command instead.

Here is a trivial example
\begin{lstlisting}[language=Stata]
clear
use "exampw1.dta", clear
tempfile temp1    // create a temporary file
save "`temp1'"      // save memory into the temporary file
use "exampw2.dta", clear
merge 1:1 v001 v002 v003 using "`temp1'" // use the temporary file
\end{lstlisting}

\subsubsection{Evaluating inequalities}
In Stata, missing numeric observations take the value of positive infinity.
Thus, when evaluating inequalities, $.>X$ is true for any $X$.
I prefer using the command \texttt{inrange( )} to avoid this issue.
Consider the following two ways of generating a dummy indicating that x1 takes a positive value:
\begin{lstlisting}[language=Stata]
gen byte positive1 = 1 if x1>=0
gen byte positive2 = (inrange(x1,0,.)==1)
\end{lstlisting}
The first command will cause positive1 to be true if x1 is non-negative or if x1 is missing.
By contrast, the second command will set positive2 to true only if x1 is non-negative and non-missing.
