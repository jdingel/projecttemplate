If you just joined our team, welcome.
A few suggestions about getting started:
\begin{itemize}
\item
Be prepared to get stumped and make mistakes.
Ask the coauthors and RAs for help.
We've all been there.
Do not struggle alone in silence.
\item
An existing project is a trove of material demonstrating how we work in practice.
Ask a coauthor to add you to a repository so that you can read through others' pull request reviews.
\item
Some RAs think mastering command-line Git is better than using a GUI app:
``using the command line for Git forces you to know and understand the Git commands you issue.''
\item
An existing project is a trove of material demonstrating how we work in practice.
Read existing Makefiles to learn how Make works.
See the \href{https://github.com/jdingel/DingelNeiman-workathome}{Dingel and Neiman (2020) replication package}.
\item
ChatGPT is very good at both explaining shell scripts and composing new ones based on precise instructions.
ChatGPT is a tool that you should leverage.
You are responsible for your outputs.
\item
Try to apply our computing tools and workflow to research substance that you've already mastered.
Were you an RA? Did you write a thesis? 
Take that data and code and get it on GitHub, take your Word document and write it in \LaTeX, write a Makefile to replace your \texttt{master.R}, and so forth.
\item
We hope that one of your first assigned tasks will be ``refactoring'' code (improving code without creating new functionality).
Because the existing code already has valid inputs and outputs,
this assignment will provide a complete description of the pre-requisites and a clear benchmark by which your work will be evaluated.
Look for the ``refactoring/rewriting'' label on GitHub issues.
\item
We don't do \href{https://en.wikipedia.org/wiki/Pair_programming}{pair programming},
but you should aim to work on code live in front of other team members during your early weeks.
They'll notice shortcuts and tools you're failing to deploy as you work.
\end{itemize}
