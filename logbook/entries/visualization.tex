These are Jonathan's standards, and he's willing to argue for them.

\begin{itemize}
	\item Generally, follow Edward Tufte's \href{https://www.edwardtufte.com/tufte/books_vdqi}{\textit{Visual Display of Quantitative Information}}.
	Mostly important, maximize the \href{https://www.coursera.org/learn/python-plotting/lecture/qFnP9/graphical-heuristics-data-ink-ratio-edward-tufte}{data-ink ratio}.
	\item At minimum, let's adhere to Schwabish ``\href{https://www.aeaweb.org/articles?id=10.1257/jep.28.1.209}{An Economist's Guide to Visualizing Data}'' (\textit{JEP} 2014)
	\item Never make a graphic with fewer than a dozen data points. A dozen data points belong in a table.
	\item Please impose \texttt{graphregion(color(white))} on every \texttt{twoway} plot created in Stata.
\end{itemize}

Jonathan's projects also typically adhere to these additional guidelines:
\begin{itemize}
\item
We want our figures to be interpretable in black and white, in case
someone prints the slides or paper.
Differentiate lines and points by pattern or symbol in addition to color.
\item
Stata legends should not have borders.
Use \texttt{lcolor(white)} within the \texttt{legend()} option in \texttt{twoway} graphs
(\texttt{lstyle(none)} leaves a light gray box).
\item
Use \texttt{bottomrule}, \texttt{toprule}, and \texttt{midrule} from the \texttt{booktabs} package
instead of \texttt{hline} to make \LaTeX\:tables look nicer.
\item
We may want different notes in the slides, paper, and logbook for the same figure or table.
Therefore, write the notes for an exhibit in the \LaTeX\:file that contains it
(or, at minimum, to a separate \LaTeX\:file), not directly onto the exhibit.
\item
Similarly, do not make the title of an exhibit part of the image itself.
We write them in the \LaTeX\:files instead.
\item
A figure that is easy to read in the paper may be annoying to read
when you are in the back of a long room watching a presentation on a projector.
For this reason, we sometimes generate a slides version of a figure with larger text and/or symbols.
\item
In general, err on the side of making the text on axis labels larger for all versions of figures.
It is rare to overshoot and have text that is too large.
\end{itemize}
